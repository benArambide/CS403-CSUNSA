% Plantilla plan de tesis Ciencia de la Computacón UNSA
% creado por
% Jose L. Sotomayor Jose L. Valencia
%

\documentclass[a4paper,11pt]{article}

\usepackage[utf8]{inputenc}
\usepackage{amsmath}
\usepackage[spanish]{babel}
\usepackage{fontenc}
\usepackage{graphicx}
\usepackage{lipsum}

\usepackage{hyperref}

\makeatletter
\renewcommand\lips@dolipsum{%
  \ifnum\value{lips@count}<\lips@max\relax
    \addtocounter{lips@count}{1}%
    \csname lipsum@\romannumeral\c@lips@count\endcsname
    \lips@dolipsum
  \fi
}
\makeatother

\title{Plan de tesis}

\begin{document}
 
 \maketitle
 
 \section{Titulo Proyecto de Tesis}
\begin{itemize}
	\item Sintetizar el problema y describir de la mejor manera el trabajo a realizarse
	\item Usar adecuadamente el lenguaje técnico profesional
	\item Estar redactado correctamente y de tal modo que otorgue al trabajo prestancia técnica y académica.
\end{itemize}

 \section{Área de Especialidad}

\begin{itemize}
	\item Indicando el área de especialidad: Inteligencia Artificial, Ingenieria de Software, Redes, etc.
\end{itemize}
 
 \section{Planteamiento y Justificación del Tema}
\begin{itemize}
	\item Indicar el problema a investigar
	\item Señalar las consideraciones científicas, técnicas, económicas o sociales que justifiquen el desarrollo de la tésis
\end{itemize}
  
 \section{Objetivos}
\begin{itemize}
	\item Considerar un único objetivo general  y varios específicos 
	\item Deben describir lo que se espera lograr a la finalización del trabajo, señalando sus alcances y características finas.
	\item Los objetivos deben ser coherentes con el título del trabajo
\end{itemize}

  
 \section{Planteamiento Teórico}
\begin{itemize}
	\item Antecedentes (obligatorio) : Indicando de manera resumida los estudios realizados hasta la fecha sobre el tema de materia de la Tésis.
	\item Formulación de hipótesis y variables involucradas
\end{itemize}
  
 \section{Metodología de Trabajo}
\begin{itemize}
	\item Indicando los procedimientos y metodología a seguir. Ademas indicar las limitaciones y restricciones al tema de tésis.

\end{itemize}
  
 \section{Índice Tentativo de la Tesis}
  \lipsum[1]
  
 \section{Cronograma de Actividades}
  \lipsum[1]
  
 %Usar BibTex
 \section{Bibliografía}
 
 
\end{document}
